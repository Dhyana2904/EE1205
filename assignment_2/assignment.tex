%\iffalse
\let\negmedspace\undefined
\let\negthickspace\undefined
\documentclass[journal,12pt,twocolumn]{IEEEtran}
\usepackage{cite}
\usepackage{amsmath,amssymb,amsfonts,amsthm}
\usepackage{algorithmic}
\usepackage{graphicx}
\usepackage{textcomp}
\usepackage{xcolor}
\usepackage{txfonts}
\usepackage{listings}
\usepackage{enumitem}
\usepackage{mathtools}
\usepackage{gensymb}
\usepackage{comment}
\usepackage[breaklinks=true]{hyperref}
\usepackage{tkz-euclide} 
\usepackage{listings}
\usepackage{gvv}                                        
\def\inputGnumericTable{}                                 
\usepackage[latin1]{inputenc}                                
\usepackage{color}                                            
\usepackage{array}                                            
\usepackage{longtable}                                       
\usepackage{calc}                                             
\usepackage{multirow}                                         
\usepackage{hhline}                                           
\usepackage{ifthen}                                           
\usepackage{lscape}

\newtheorem{theorem}{Theorem}[section]
\newtheorem{problem}{Problem}
\newtheorem{proposition}{Proposition}[section]
\newtheorem{lemma}{Lemma}[section]
\newtheorem{corollary}[theorem]{Corollary}
\newtheorem{example}{Example}[section]
\newtheorem{definition}[problem]{Definition}
\newcommand{\BEQA}{\begin{eqnarray}}
 \newcommand{\EEQA}{\end{eqnarray}}
\newcommand{\define}{\stackrel{\triangle}{=}}
\theoremstyle{remark}
\newtheorem{rem}{Remark}
\begin{document}
 \bibliographystyle{IEEEtran}
 \vspace{3cm}
 \title{\textbf{11.14-4}}
 \author{EE23BTECH11048-Ponugumati Venkata Chanakya$^{*}$% <-this % stops a space
 }
 \maketitle
 \newpage
 \bigskip
 \renewcommand{\thefigure}{\theenumi}
 \renewcommand{\thetable}{\theenumi}
 \textbf{QUESTION:}
 The 17th term of ap exceeds its 10th term by 7. FInd its common difference?\\
 \solution

 \begin{align}
     x(n) &= \{x(0)+nd\}u(n) \\
     x(17)-x(10) &= 7\\
    \implies {x(0)+17d}-{x(0)+10d} &= 7\\
    \implies d &= 1
 \end{align}

 
 \begin{table}[!ht]
    \centering
        \begin{tabular}{|l|c|c|c|c|c|}
  \hline
  &\textbf{Function}&\textbf{Periodic } &\textbf{Simple harmonic motion}&\textbf{Non Periodic}&\textbf{Period} \\\hline
  (a) & $sin(\omega t)-cos(wt)$ & Yes & Yes & No & $\frac{2\pi}{\omega}$ \\\hline
  (b) & $sin^3(\omega t)$  & Yes & Yes & No & $\frac{2\pi}{\omega}$\\\hline
  (c) & $3cos\brak{\frac{\pi}{4}-2\omega t}$ & Yes & Yes & No & $\frac{\pi}{\omega}$\\\hline
  (d) & $cos(\omega t)+cos(3\omega t)+cos(5\omega t)$ & Yes & Yes & No & $\frac{2\pi}{\omega}$\\\hline
  (e) & $exp^{\brak{-\omega^2t^2}}$ & No & No & Yes & $-$ \\\hline
  (f) &$ 1+\omega t+\omega ^2t^2$ & No & No & Yes & $-$ \\\hline
 \end{tabular}

    \caption{input parameters}
    \label{tab:10_5_3_12}
\end{table}
Let \\
\begin{align}
x(n)&= \lbrace 101,102,103,...\rbrace 
\end{align}
\begin{figure}[h!]
    \centering
    \includegraphics[width=0.4\textwidth]{figs/fig1.png}
    \caption{x(n)}
\end{figure}
 
 \end{document}
