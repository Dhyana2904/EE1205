%\iffalse
\let\negmedspace\undefined
\let\negthickspace\undefined
\documentclass[journal,12pt,onecolumn]{IEEEtran}
\usepackage{cite}
\usepackage{amsmath,amssymb,amsfonts,amsthm}
\usepackage{algorithmic}
\usepackage{graphicx}
\usepackage{textcomp}
\usepackage{xcolor}
\usepackage{txfonts}
\usepackage{listings}
\usepackage{enumitem}
\usepackage{mathtools}
\usepackage{gensymb}
\usepackage{comment}
\usepackage[breaklinks=true]{hyperref}
\usepackage{tkz-euclide} 
\usepackage{listings}
\usepackage{gvv}                                        
\def\inputGnumericTable{}                                 
\usepackage[latin1]{inputenc}                                
\usepackage{color}                                            
\usepackage{array}                                            
\usepackage{longtable}                                       
\usepackage{calc}                                             
\usepackage{multirow}                                         
\usepackage{hhline}                   \usepackage{circuitikz}                        
\usepackage{ifthen}                                           
\usepackage{lscape}

\newtheorem{theorem}{Theorem}[section]
\newtheorem{problem}{Problem}
\newtheorem{proposition}{Proposition}[section]
\newtheorem{lemma}{Lemma}[section]
\newtheorem{corollary}[theorem]{Corollary}
\newtheorem{example}{Example}[section]
\newtheorem{definition}[problem]{Definition}
\newcommand{\BEQA}{\begin{eqnarray}}
 \newcommand{\EEQA}{\end{eqnarray}}
\newcommand{\define}{\stackrel{\triangle}{=}}
\theoremstyle{remark}
\newtheorem{rem}{Remark}
\begin{document}
 \bibliographystyle{IEEEtran}
 \vspace{3cm}
 \title{\textbf{ME 36}}
 \author{EE23BTECH11048-Ponugumati Venkata Chanakya$^{*}$% <-this % stops a space
 }
 \maketitle

 \bigskip
 \renewcommand{\thefigure}{\theenumi}
 \renewcommand{\thetable}{\theenumi}
 \textbf{QUESTION:}
In the circuit shown below, $R_1=2\ohm$,$R_2=1\ohm$,$L_1=2$ h,and $L_2=0.5$ H. Which of the following describe(s) the correct characteristics of the circuit ?\\
\begin{center}
\begin{circuitikz}
   \draw (0,0)
   to[sV, v=$V_s$] (0,2) % Sinusoidal voltage source
   to[L, l=$L_1$] (2,2)   % Inductor
   to[L, l=$L_2$] (4,2)   % Resistor
   to [european][R, l=$R_2$] (4,0)   % Resistor
   -- (0,0)              % Connection to ground
   (2,2) to [european][R, l=$R_1$] (2,0);  % Resistor
   \draw (4,2) to[short, *-o] (5,2) node[right] {};
   \draw (4,0) to[short, *-o] (5,0) node[right] {};
   \draw (5,1) node[right]{$v_0$};
\end{circuitikz}
\end{center}
\begin{enumerate}
    \item Second order high pass filter \\
    \item Second order low pass filter\\
    \item Under damped system \\
    \item Overdamped system\\
\end{enumerate}
\solution\\
Converting above circuit to frequency domain using laplace transform\\
let $V_1$ and $V_2$ be voltages at shown positions\\
\begin{center}
\begin{circuitikz}
   \draw (0,0)
   to[sV, v=$V_s$] (0,2) % Sinusoidal voltage source
   to[L, l=$sL_1$] (2,2)   % Inductor
   to[L, l=$sL_2$] (4,2)   % Resistor
   to [european][R, l=$R_2$] (4,0)   % Resistor
   -- (0,0)              % Connection to ground
   (2,2) to [european][R, l=$R_1$] (2,0);  % Resistor
   \draw (4,2) to[short, *-o] (5,2) node[right] {};
   \draw (4,0) to[short, *-o] (5,0) node[right] {};
   \draw (5,1) node[right]{$v_0$};
   \node at (2,2.3){$V_1$};
   \node at (4,2.3){$V_2$};
\end{circuitikz}
\end{center}
 \begin{table}[!ht]
    \centering
         \begin{tabular}{|c|c|} 
      \hline
\textbf{Variable}& \textbf{Value}\\\hline
         $R_1$ & $2\ohm$\\\hline
          $R_2$ &$1\ohm$\\\hline
          $L_1$  &$2$ H \\ \hline
         $L_2$  &$0.5$ H \\ \hline
    \end{tabular}

    \caption{input parameters}
\end{table}
\begin{align}
    V_0&=V_1\brak{\frac{R_2}{R_2+sL_2}}\\
   V_1&=V_s\brak{\frac{R_1\brak{\frac{sL_2+R_2}{R_1+R_2+SL_2}}}{sL_1+R_1\brak{\frac{sL_2+R_2}{R_1+R_2+SL_2}}}}\\
   V_1&=V_s\brak{\frac{2+s}{(2+s)+s(6+s)}}\\
   V_0&=V_s\brak{\frac{2}{s^2+7s+2}}\\ \label{BM_43.4}
   \text{let } s&=j\omega\\
   V_0&= V_s\brak{\frac{2}{-\omega^2 + 7\jmath\omega+2}}\\
   &=V_s\brak{\frac{4-2\omega^2-7\jmath\omega}{\omega^4+45\omega^2+4}}
\end{align}
For lower frequency $V_0$ is finite and for higher frequency $V_0$ is zero\\
 $\therefore $  Second order low pass filter\\
 From \ref{BM_43.4}
\begin{align}
    s^2+7s+2&=0\\
    \text{ for } as^2+bs+c=0\\
\zeta\text{(Damping Factor)}&=\frac{b}{2\sqrt{ac}}\\
\text{By comparing }\zeta&=\frac{7}{2\sqrt{2}}\\
\implies \zeta>1
\end{align}
$\therefore $  Over-damped System\\
$\therefore$ B,D are correct options
\end{document}
