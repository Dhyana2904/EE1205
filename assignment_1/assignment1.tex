%\iffalse
\let\negmedspace\undefined
\let\negthickspace\undefined
\documentclass[journal,12pt,twocolumn]{IEEEtran}
\usepackage{cite}
\usepackage{amsmath,amssymb,amsfonts,amsthm}
\usepackage{algorithmic}
\usepackage{graphicx}
\usepackage{textcomp}
\usepackage{xcolor}
\usepackage{txfonts}
\usepackage{listings}
\usepackage{enumitem}
\usepackage{mathtools}
\usepackage{gensymb}
\usepackage{comment}
\usepackage[breaklinks=true]{hyperref}
\usepackage{tkz-euclide} 
\usepackage{listings}
\usepackage{gvv}                                        
\def\inputGnumericTable{}                                 
\usepackage[latin1]{inputenc}                                
\usepackage{color}                                            
\usepackage{array}                                            
\usepackage{longtable}                                       
\usepackage{calc}                                             
\usepackage{multirow}                                         
\usepackage{hhline}                                           
\usepackage{ifthen}                                           
\usepackage{lscape}

\newtheorem{theorem}{Theorem}[section]
\newtheorem{problem}{Problem}
\newtheorem{proposition}{Proposition}[section]
\newtheorem{lemma}{Lemma}[section]
\newtheorem{corollary}[theorem]{Corollary}
\newtheorem{example}{Example}[section]
\newtheorem{definition}[problem]{Definition}
\newcommand{\BEQA}{\begin{eqnarray}}
 \newcommand{\EEQA}{\end{eqnarray}}
\newcommand{\define}{\stackrel{\triangle}{=}}
\theoremstyle{remark}
\newtheorem{rem}{Remark}
\begin{document}
 \bibliographystyle{IEEEtran}
 \vspace{3cm}
 \title{\textbf{11.14-4}}
 \author{EE23BTECH11048-Ponugumati Venkata Chanakya$^{*}$% <-this % stops a space
 }
 \maketitle
 \newpage
 \bigskip
 \renewcommand{\thefigure}{\theenumi}
 \renewcommand{\thetable}{\theenumi}
 \textbf{QUESTION:}
 
 Which of the following functions of time represent (a) simple harmonic, (b) periodic
 but not simple harmonic, and (c) non-periodic motion? Give period for each case of
 peiodic motion ($\omega$ is any positive constant):\\
 (a)$\sin\brak{\omega t}-\cos\brak{\omega t}$\\
 (b)$\sin^3\brak{\omega t}$\\
 (c)$3\cos\brak{\frac{\pi}{4}- 2\omega t}$\\
 (d)$\cos\brak{\omega t}+\cos\brak{3\omega t}+\cos\brak{5\omega t}$\\
 (e)$exp(-\omega^2 t^2)$\\
 (f)$1+\omega t+\omega^2 t^2$\\
 Answer:\\
   Definition of period:\\The period is denoted by the symbol "T," and it represents the time interval required for the motion to go through one complete cycle\\
   \\
   \begin{enumerate}
 \item $\sin(\omega t)- \cos(\omega t)$\\
  \end{enumerate}
 \begin{align}
  &= \sin(\omega t) - \sin\brak{\frac{\pi}{2} - \omega t}\\
  &=2 \cos\brak{\frac{\pi}{4}} \sin\brak{\omega t - \frac{\pi}{4}}\\
  &=\sqrt{2}\sin\brak{\omega t - \frac{\pi}{4}}
 \end{align}
 $\therefore$ Simple harmonic motion with period {T} =$\frac{2\pi}{\omega}$\\ Phase angle of $\brak{\frac{-\pi}{4}}$ or$\brak{\frac{7\pi}{4}}$\\
\\
\begin{enumerate}
 \setcounter{enumi}{1} 
    \item $\sin^3(wt)$\\
\end{enumerate}
 
 \begin{align}
  &=\frac{1}{4}(3\sin(\omega t)-\sin(3 \omega t))
 \end{align}
 $\therefore$  Simple harmonic motion with period {T} = $\frac{2\pi}{\omega}$ \\
  phase angle zero\\
\\
\begin{enumerate}
 \setcounter{enumi}{2} 
    \item $3\cos\brak{\frac{\pi}{4}-2\omega t}$\\
\end{enumerate}
 \begin{align}
  =3\cos\brak{2\omega t-\frac{\pi}{4}}\\
 \end{align}
 Simple harmonic motion with period {T} = $\frac{\pi}{\omega}$  and a phase angle of  $\brak{\frac{-\pi}{4}}$ or$\brak{\frac{7\pi}{4}}$\\
 \\
 \begin{enumerate}
 \setcounter{enumi}{3} 
    \item  $\cos(\omega t)+\cos(3\omega t)+\cos(5\omega t)$\\
\end{enumerate}
 \begin{align}
  &=\cos(\omega t)+\cos(5\omega t)+\cos(3\omega t)\\
  &=2\cos\brak{\frac{\omega t+5\omega t}{2}}\cos\brak{\frac{5\omega t-\omega t}{2}} +\cos(3\omega t)\\
  &=2\cos(3\omega t) \cos(\omega t)+\cos(3\omega t)\\
  &=\cos(3\omega t) (1+2\cos(\omega t))
 \end{align}
 Period of $\cos(3\omega t)$ is $\frac{2\pi}{3\omega}$\\ 
 Period of $1+2\cos(\omega t)$ is $\frac{2\pi}{\omega}$\\ 
 Lcm is $\frac{2\pi}{\omega}$\\
 $\therefore$  Simple harmonic motion with period $\frac{2\pi}{\omega}$\\
 \\
 \begin{enumerate}
 \setcounter{enumi}{4} 
    \item  $exp^{\brak{-\omega^2t^2}}$\\
\end{enumerate}
       \begin{center}
     As $T\to\infty$\\
    $exp\brak{^{-\omega^2t^2}} \to \infty$\\ 
       \end{center}
    $\therefore$  This never repeats and non periodic\\
    \\
    \begin{enumerate}
 \setcounter{enumi}{5} 
    \item $1+\omega t+\omega^2t^2$\\
\end{enumerate}
 \begin{center}
  As $T\to\infty$\\
  $1+\omega t+\omega^2t^2  \to \infty$\\
  \end{center}
  $\therefore$ This never repeats and non periodic\\
 \newpage
 
 \renewcommand{\thefigure}{\theenumi}
 \renewcommand{\thetable}{\theenumi}
 
 
 \begin{figure}[h!]
    \centering
    \includegraphics[width=0.4\textwidth]{figs/a1_fig1.png}
    \caption{$\sin(\omega t)- \cos(\omega t)$}
\end{figure}
 \begin{figure}[h!]
    \centering
    \includegraphics[width=0.4\textwidth]{figs/a1_fig2.png}
    \caption{$\sin^3(\omega t)$}
\end{figure}
\begin{figure}[h!]
    \centering
    \includegraphics[width=0.4\textwidth]{figs/a1_fig3.png}
    \caption{$3\cos\brak{\frac{\pi}{4}- 2\omega t}$}
\end{figure}
\begin{figure}[h!]
    \centering
    \includegraphics[width=0.4\textwidth]{figs/a1_fig4.png}
    \caption{$\cos(\omega t)+\cos(3\omega t)+\cos(5\omega t)$}
\end{figure}
\begin{figure}[h!]
    \centering
    \includegraphics[width=0.4\textwidth]{figs/a1_fig5.png}
    \caption{$exp^{\brak{-\omega^2 t^2}}$}
\end{figure}
\begin{figure}[h!]
    \centering
    \includegraphics[width=0.4\textwidth]{figs/a1_fig6.png}
    \caption{$1+\omega t+\omega^2 t^2$}
\end{figure}
\newpage
\begin{flushleft}
  \begin{table}[h]
   \def\arraystretch{1.5}:
   \caption{Summary}
   \label{tab:table.11.14-4}
   \begin{tabular}{|l|c|c|c|c|c|}
  \hline
  &\textbf{Function}&\textbf{Periodic } &\textbf{Simple harmonic motion}&\textbf{Non Periodic}&\textbf{Period} \\\hline
  (a) & $sin(\omega t)-cos(wt)$ & Yes & Yes & No & $\frac{2\pi}{\omega}$ \\\hline
  (b) & $sin^3(\omega t)$  & Yes & Yes & No & $\frac{2\pi}{\omega}$\\\hline
  (c) & $3cos\brak{\frac{\pi}{4}-2\omega t}$ & Yes & Yes & No & $\frac{\pi}{\omega}$\\\hline
  (d) & $cos(\omega t)+cos(3\omega t)+cos(5\omega t)$ & Yes & Yes & No & $\frac{2\pi}{\omega}$\\\hline
  (e) & $exp^{\brak{-\omega^2t^2}}$ & No & No & Yes & $-$ \\\hline
  (f) &$ 1+\omega t+\omega ^2t^2$ & No & No & Yes & $-$ \\\hline
 \end{tabular}

  \end{table}
 \end{flushleft}
\end{document} 
